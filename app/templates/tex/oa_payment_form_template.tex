\documentclass[9pt,a4paper]{extreport}
\usepackage[UTF8]{ctex}
\usepackage{tabularx}
\usepackage{fontspec}
\usepackage{ulem}
\usepackage{makecell}
\usepackage{adjustbox}
\usepackage{amssymb}
\usepackage{pifont}
\usepackage{array}
\usepackage{calc}
\usepackage{ifthen}
\usepackage{datetime2}
\usepackage[colorlinks=true,linkcolor=black,urlcolor=blue]{hyperref}

\renewcommand{\ULdepth}{1.8pt}
\renewcommand{\arraystretch}{1.2}
\newcommand{\notefont}{\fontsize{7pt}{1\baselineskip}\selectfont}
\newcommand{\cellfont}{\fontsize{8pt}{1\baselineskip}\selectfont}
\newcommand{\emphisefont}{\fontsize{9pt}{1.5\baselineskip}\selectfont}
\newcommand{\emphbold}[1]{\textbf{\cellfont #1}}
\DTMsetdatestyle{iso}  % 设置为 ISO 格式 (YYYY-MM-DD)


\thispagestyle{empty}

% 字体设置
\setmainfont[Path=((data["fonts_dir"])),
             Extension=.ttf,
             UprightFont=*regular,
             BoldFont=*bold,
             ItalicFont=*italic,]{arial_}
\setCJKmainfont[Path=((data["fonts_dir"])),
             Extension=.ttc,
             UprightFont=*regular,
             BoldFont=*bold,
             ]{msyh_}


% 设置中文断行
\XeTeXlinebreaklocale "zh"
\XeTeXlinebreakskip = 0pt plus 1pt minus 0.1pt

% 定义页面边距的变量名称和值的参数
\newcommand{\pagetopmargin}{1in}
\newcommand{\pagerightmargin}{15mm}
\newcommand{\pageleftmargin}{15mm}
\newcommand{\pagebottommargin}{1in}

% 使用参数控制页面的margin
\usepackage[top=\pagetopmargin,
            right=\pagerightmargin,
            left=\pageleftmargin,
            bottom=\pagebottommargin,
            ]{geometry}  %%%%默认页面设置参数


% 自定义标签列宽度,可自行调整
\newcolumntype{L}{>{\raggedright\arraybackslash}X}
\setlength{\tabcolsep}{3pt} % 控制列左右间隔,越小越贴边
\newcommand{\longrule}{\uline{\hspace{\fill}}}
% 自定义下划线命令
\newcommand{\fulluline}[1]{%
    \makebox[0pt][l]{\uline{\hspace{\linewidth}}}#1%
}

\newcommand{\colline}[1]{%
  \makebox[0pt][l]{\uline{\hspace{0.1\columnwidth}}}#1%
}

% 定义变量
\newcommand{\banksettled}{((data["bank_settled"]))}
\newcommand{\chequeno}{((data["cheque_no"]))}
\newcommand{\otherpaymethod}{}

\listfiles


\begin{document}
%%% 报告标题,其中报告标题应该输入参数:报告类型、报告期
\begin{center}
\textbf{\LARGE{PAYMENT REQUESTS FORM} \\ \vspace{1mm} \LARGE{付款申请单}}
\end{center}

\vspace{8mm}
\noindent
\begin{tabularx}{\textwidth}{@{}l X @{\hspace{15pt}} l p{0.21\textwidth} @{}}
COMPANY 公司:    & \fulluline{\hspace{0.25em}\textbf{((data["payer"]|elatex))}}
& PRF No. 申请单编号: & \fulluline{\hspace{0.25em}\textbf{((data["order_id"]|elatex))}}  \\
DEPARTMENT 部门: & \fulluline{\hspace{0.25em}\textbf{((data["department"]|elatex))}}
& DATE 日期:         & \fulluline{\hspace{0.25em}\textbf{((data["prf_date"]))}}  \\
PROJECT 项目:    & \fulluline{\hspace{0.25em}\textbf{((data["project_name"]|elatex))}}
& PROJECT CODE 项号: & \fulluline{\hspace{0.25em}\textbf{((data["project_code"]|elatex))}}  \\
\end{tabularx}

\vspace{2mm}
\noindent
\begin{table}[h]
\renewcommand{\arraystretch}{1.5}
\begin{tabularx}{\textwidth}{|@{\hspace{1em}}>{\raggedright\arraybackslash}p{0.55\textwidth}|@{\hspace{1em}}p{0.18\textwidth}|@{\hspace{1em}}X|}
\multicolumn{3}{l}{DETAILS OF PAYMENT 付款明细} \\
\hline
Name of Payee 收款人姓名: \newline \textbf{((data["payee"]|elatex))} & Currency 货币: \newline \textbf{((data["currency"]|elatex))} & Amount 金额: \newline \textbf{((data["amount"]))} \\
\hline
 \parbox[t]{0.5\textwidth}{Payment Reason 付款理由: \\[5em]  \textbf{((data["payment_reason"]|elatex))}}
  & \multicolumn{2}{l|} {
    \renewcommand{\arraystretch}{2}
    \begin{tabular}[t]{p{4cm}l}  % 修改这里:使用 [t] 顶部对齐,p{4cm} 确保不换行
    Payment Method 付款方式:  & \\
    & \notefont{\textit{DBS Cheque No.}}  \\
    % 第1行 Cheque
    \ifthenelse{\equal{\banksettled}{Cheque}}%
        {\ding{51}\phantom{xx}Cheque 支票  & \colline{\notefont\textit\chequeno} \\}%
        {$\Box$\phantom{xx}Cheque 支票  & \colline \\}
    % 第2行 Online
    \ifthenelse{\equal{\banksettled}{Online}}%
        {\ding{51}\phantom{xx}Online 网上 & \\}
        {$\Box$\phantom{xx}Online 网上 & \\}
    % 第3行 Others
    \ifthenelse{\equal{\banksettled}{Others}}%
    {\ding{51}\phantom{xx}Others 其它 & \colline{\notefont\textit\otherpaymethod} \\}
    {$\Box$\phantom{xx}Others 其它 & \colline \\}
    \vspace{1em}
    \end{tabular}
} \\
\hline
\renewcommand{\arraystretch}{1.1}
\begin{tabular}[t]{p{0.5\textwidth}}  % 修改这里:使用 p{width} 替代 l
  Supporting Document Attached 附上支持文件: \\
  (% for attachment in data["attachments"] %)
  (% if attachment|elatex %)
  \emphbold{((attachment|elatex))}(% if not loop.last %) \\(% endif %)
  (% else %)
  (% endif %)
  (% endfor %)
  (% if not data["attachments"] %)
  (% endif %)
\end{tabular}
 & \multicolumn{2}{l|}{
 \renewcommand{\arraystretch}{1.1}
  \begin{tabular}[t]{p{0.3\textwidth}}  % 添加 [t] 参数,并使用 p{width}
  Payment Date 付款日期: \\
  \textbf{((data["prf_date"]))} \\
  \phantom{xx}
  \end{tabular}
} \\
\hline
\end{tabularx}
\end{table}

% APPROVAL 审批部分
\vspace{2mm}
\noindent
\begin{table}[h]
\newcolumntype{Y}{>{\centering\arraybackslash}X}
\newcommand{\signbox}{\rule{0pt}{50pt}} % 签名框高度
\newcommand{\datetext}[1]{{\raggedright\notefont{#1}}} % 日期文 字格式
\newcommand{\dateformatf}[1]{\multicolumn{1}{|l|}{\notefont{#1}}}
\newcommand{\dateformat}[1]{\multicolumn{1}{l|}{\notefont{#1}}}
\renewcommand{\arraystretch}{1.2}
\begin{tabularx}{\textwidth}{|Y|Y|Y|Y|Y|}
\multicolumn{5}{l}{APPROVAL 审批} \\
\hline
Initiator & Reviewed By & Department & Finance & Group CFO \\
经办人 & 财务审阅 & 部门负责人 & 财务负责人 & 首席财务官 \\
\hline
\signbox & \signbox & \signbox & \signbox & \signbox \\
\dateformatf{Date 日期:\DTMtoday} & \dateformat{Date 日期:} & \dateformat{Date 日期:} & \dateformat{Date 日期:} & \dateformat{Date 日期:}  \\
\hline
\end{tabularx}
\end{table}

% OA申请单号
\vspace{2mm}
    \noindent
\begin{tabularx}{\textwidth}{|@{\hspace{1em}}>{\raggedright\arraybackslash}p{0.2\textwidth}|@{\hspace{1em}}>{\raggedright\arraybackslash}X|}
    \multicolumn{2}{l}{OA申请单号} \\
    \hline
    OA申请单号: & \href{((data["oa_url"]|elatex))}{((data["oa_no"]|elatex))} \\
    \hline
\end{tabularx}

\end{document}
